\documentclass{book}


% pdfx will set color profile etc. information appropriately, so the pdf renders
% consistently across devices. But, it doesn't work with the xelatex-based tectonics
	\usepackage[utf8]{inputenc}

%%%
% import all needed packages and macros
%%%

%%
%% All packages and macros needed for the problemsets
%%

\usepackage{amsmath}

\usepackage{lipsum}
%\usepackage{showframe}
%\usepackage{layout}


\usepackage[charter,cal=cmcal]{mathdesign} %different font
%\usepackage{avant}

\usepackage{microtype}
\usepackage{mathtools}
\usepackage{etoolbox}
%\usepackage{amsfonts}
%\usepackage{amssymb}
\usepackage{graphicx}
\graphicspath{{images/}}
\usepackage[inline]{enumitem}
\usepackage{xparse}
\usepackage{ifthen}
\usepackage{caption}
\usepackage{subcaption}
\usepackage{tikz}
	\usetikzlibrary{fit}
	\usetikzlibrary{fadings}
	\usetikzlibrary{calc}
	\tikzset{>=latex}
	\usetikzlibrary{cd}
	\usetikzlibrary{spy}
	\usetikzlibrary{patterns}
\usepackage{fancyhdr}
\usepackage{calc}
\usepackage{wrapfig}
\usepackage{marginnote}
\usepackage{mparhack}
\usepackage{marginfix}
\usepackage{indextools}
\usepackage[open=false]{bookmark}  % render the pdf TOC in the proper order
\hypersetup{
	hidelinks=true,
	linkcolor = {0 0 1},
	unicode=true,
	psdextra=true,
}
\usepackage[margin=.5in, lmargin=1in, rmargin=1in]{geometry}
\usepackage{tabularx}
\usepackage{qrcode}


%\usepackage[
%  linktocpage=false,      % no page numbers are clickable
%  colorlinks=false,       % no color
%  breaklinks=true,        % break URLs
%  bookmarks,              % creates bookmarks in pdf
%  hyperfootnotes=true,    % clickable footnotes
%  pdfborder={0 0 0},      % for removing borders around links
%  bookmarksnumbered=true, % If Acrobat bookmarks are requested, include section numbers.
%  bookmarksopen=false,    % If Acrobat bookmarks are requested, show them with all the subtrees expanded.
%  %hidelinks=true,
%  %linkcolor=blue,
%  %citecolor=blue,
%  %urlcolor=blue,
%  pdfpagemode={UseOutlines}, % show pdf bookmarks (indices) on startup; does not function all the time
%  pdftitle={...}, % title
%  pdfauthor={...}, % author
%  pdfkeywords={...}, % subject of the document
%  pdfsubject={...}, % list of keywords
%  pdfmenubar=true,        % make PDF viewer’s menu bar visible
%  pdfpagelabels,
%]{hyperref}
%\usepackage[hidelinks,]{hyperref}
\usepackage{fnpct} % fancy footnote spacing
\usepackage{bm}
\usepackage{systeme}
\usepackage{datatool}% http://ctan.org/pkg/datatool for sorted lists
\usepackage{xspace}
\usepackage{multicol}

\usepackage{pgfplots}
\pgfplotsset{compat=1.17}
	\usepgfplotslibrary{fillbetween}



\newcommand{\addtoolboxitem}[4]{\subsubsection*{#1}\vspace{-.1cm} #2 \vspace{-.5cm}\paragraph{Variants:} #3 \vspace{-.5cm}\paragraph{Uses:} #4\par\noindent\rule{\linewidth}{0.4pt}}
\newcommand{\enquote}[1]{``#1''}
%%%
% Useful Linear Algebra macros
%%%
\newcommand{\declarecommand}[1]{\providecommand{#1}{}\renewcommand{#1}}
\declarecommand{\R}{\mathbb{R}}  % we don't care if it's already defined.  We really want *this* command!
\declarecommand{\Z}{\mathbb{Z}}
\declarecommand{\Q}{\mathbb{Q}}
\declarecommand{\N}{\mathbb{N}}
\declarecommand{\C}{\mathbb{C}}
\declarecommand{\d}{\mathrm{d}}
\declarecommand{\dd}{\mathbbm{d}} % exterior derivative
\DeclareMathOperator{\Span}{span}
\DeclareMathOperator{\Img}{img}
\DeclareMathOperator{\Id}{id}
\DeclareMathOperator{\Ident}{\Id}
\DeclareMathOperator{\Vol}{Vol}
\DeclareMathOperator{\VolChange}{Vol\hspace{1.5pt}Change}
\DeclareMathOperator{\Range}{range}
\DeclareMathOperator{\Rref}{rref}
\DeclareMathOperator{\Rank}{rank}
\DeclareMathOperator{\Comp}{\Vcomp}
\DeclareMathOperator{\Vcomp}{v\hspace{1pt}comp}
\DeclareMathOperator{\Null}{null}
\DeclareMathOperator{\Nullity}{nullity}
\DeclareMathOperator{\Char}{char}
\DeclareMathOperator{\Proj}{proj}
\DeclareMathOperator{\Flux}{Flux}
\DeclareMathOperator{\Circ}{Circ}
\DeclareMathOperator{\chr}{char}
\DeclareMathOperator{\Dim}{dim}
\DeclareMathOperator{\Perp}{perp}
\DeclareMathOperator{\Ker}{kernel}
\DeclareMathOperator{\Row}{row}
\DeclareMathOperator{\Col}{col}
\DeclareMathOperator{\Rep}{Rep}
\newcommand{\BasisChange}[2]{[#2\!\leftarrow\!#1]}
\newcommand{\proj}{\Proj}
\newcommand{\rref}{\Rref}
\newcommand{\xhat}{{\vec e_1}}
\newcommand{\yhat}{{\vec e_2}}
\newcommand{\zhat}{{\vec e_3}}
\newcommand{\sbasis}[1]{\vec { e}_{#1}}
\newcommand{\mat}[1]{\begin{bmatrix*}[r]#1\end{bmatrix*}}
\newcommand{\matc}[1]{\begin{bmatrix}#1\end{bmatrix}}
\newcommand{\formarg}[2]{\big(#1;\, #2\big)}
\DeclarePairedDelimiter\abs{\lvert}{\rvert}
\DeclarePairedDelimiter\Abs{\lvert}{\rvert}
\DeclarePairedDelimiter\norm{\lVert}{\rVert}
\newcommand{\Norm}[1]{\norm{#1}}
% just to make sure it exists
\providecommand\given{}
% can be useful to refer to this outside \Set
\newcommand\SetSymbol[1][]{%
	\nonscript\::%
	\allowbreak
	\nonscript\:
	\mathopen{}}
\DeclarePairedDelimiterX\Set[1]\{\}{%
	\renewcommand\given{\SetSymbol[\delimsize]}
	#1
}
\newcommand{\Rrefto}{\xrightarrow{\text{row reduces to}}}


% redefine bmatrix,etc to allow optional argument for augmenting
% code from https://tex.stackexchange.com/questions/2233/whats-the-best-way-make-an-augmented-coefficient-matrix
\makeatletter
\renewcommand*\env@matrix[1][*\c@MaxMatrixCols c]{%
  \hskip -\arraycolsep
  \let\@ifnextchar\new@ifnextchar
  \array{#1}}
\makeatother

\newcommand{\scaledgrid}[1]{%
	\begin{tikzpicture}[scale=#1]
		\draw[thin, white!20!black, dotted] (-4.1,-4.1) grid (4.1,4.1);
		\draw[ <->] (-4.3,0) -- (4.3,0);
		\draw[ <->] (0,-4.3) -- (0,4.3);
	\end{tikzpicture}
}
\newcommand{\scaledshortgrid}[1]{%
	\begin{tikzpicture}[scale=#1]
		\draw[thin, white!20!black, dotted] (-4.1,-2.1) grid (4.1,2.1);
		\draw[ <->] (-4.3,0) -- (4.3,0);
		\draw[ <->] (0,-2.3) -- (0,2.3);
	\end{tikzpicture}
}
\newcommand{\singlegrid}{\scaledgrid{1}}
\newcommand{\doublegrid}{\mbox{\scaledgrid{.9}\scaledgrid{.9}}\par}
\newcommand{\triplegrid}{\mbox{\scaledgrid{.6}\scaledgrid{.6}\scaledgrid{.6}}\par}

% labels for source attributions
\NewDocumentCommand{\beezer}{o}{%
	\IfNoValueTF{#1}{%
		{\color{blue}\sffamily{B}}%
	}{%
		{\color{blue}\sffamily{B}}%  XXX Todo, make this href to the appropriate problem number
	}\xspace%
}
\NewDocumentCommand{\hefferon}{o}{%
	\IfNoValueTF{#1}{%
		{\color{blue}\sffamily{H}}%
	}{%
		{\color{blue}\sffamily{H}}%  XXX Todo, make this href to the appropriate problem number
	}\xspace%
}

\DeclareDocumentEnvironment{beforeyouread}{}{
Before you read, make sure you are comfortable with the following.
Please do the ``Quick Check'' problem to see if you are comfortable with each
task.
\begin{itemize}
}{
\end{itemize}
}
\newcommand\quickcheck[1]{\par {\footnotesize \textsc{Quick Check.} \textrm{#1}}}

% Dummy, voidable environments
\DeclareDocumentEnvironment{bookonly}{o}{}{}
\DeclareDocumentEnvironment{displayonly}{o}{}{}



%%
% Allow hiding of environments
%%
\usepackage{environ}% http://ctan.org/pkg/environ



		\newcolumntype{s}{>{\hsize=.5\hsize}X}

\begin{document}
%%
%% Import definitions from definition.tex; all definitions can be restated multiple times
%%


%%
%% End Definitions
%%



\renewcommand{\headrulewidth}{0pt}
\pagestyle{fancy}
\section*{Lesson Planning}

This worksheet is designed to guide you through the lesson-planning processes. Creating a good lesson
plan takes a lot of work, but many of the steps will become second-nature.

\begin{enumerate}
	\item \textbf{Topic.} What are the topic(s) of the lesson?

		You can use single words like ``Derivatives'' or goal-like phrases like ``learn how to use logarithms
		to compute difficult derivatives''.
	\vspace{\stretch{1}}

	\item \textbf{Context.} List some contextual factors that might
		affect your lesson? (E.g., Length of class period? Number of
		students? First day of class/day before midterm/etc.? New
		topic/building on previous topic?)
	\vspace{\stretch{1}}

	\item \textbf{Self-justification.} Write a short justification
	of why this topic matters. Consider the following questions as
	you answer: Why is this topic important to you? Why would it be
	important to the students? What can people who know this topic
	do that others cannot?
	\vspace{\stretch{2}}
	\newpage

	\item \textbf{Learning Objectives.} 
		\begin{enumerate}
			\item Create at least two learning objectives for your lesson.
				Use verbs from the list of Learning Objectives Verbs (in the Resources section at the end of this packet).

			\item Annotate each objective with (i) the context
			they should be demonstrating the skill (e.g.,
			after this lesson/on a written exam/etc.) and
			(ii) how could you measure whether a student
			has achieved the objective?

		\item[\bfseries Objective 1:] ~
			\vspace{\stretch{.5}}
		\item[\bfseries Objective 2:] ~
			\vspace{\stretch{.5}}
		\end{enumerate}

	\item \textbf{Activities Brainstorm.} List an activity you could
	do in class to achieve each objective. Annotate your activity
	with (i) how much time would it take, (ii) what you would be
	doing during the activity, (iii) what the students would be
	doing during the activity.
	\vspace{\stretch{1}}

	\newpage
	\item \textbf{Plan your lesson.}

		\begin{tabularx}{\textwidth}{|s|X|s|}
			\hline
			Date & Title & Class Duration\\
			\hline
			\rule[-.3\baselineskip]{0pt}{1.5cm}~&~&~\\
			\hline
		\end{tabularx}
		
		\begin{tabularx}{\textwidth}{|X|}
			\hline
			Objectives\\
			\hline
			\begin{itemize}
				\item ~

				\rule[-.3\baselineskip]{0pt}{1.5cm}~
				\item ~

				\rule[-.3\baselineskip]{0pt}{1.5cm}~
			\end{itemize}\\
			\hline
		\end{tabularx}
		
		\begin{tabularx}{\textwidth}{|c|X|}
			\hline
			Timestamp & Activity\\
			\hline
				\rule[-.3\baselineskip]{0pt}{3.2cm}~ & ~\\
			\hline
				\rule[-.3\baselineskip]{0pt}{3.2cm}~ & ~\\
			\hline
				\rule[-.3\baselineskip]{0pt}{3.2cm}~ & ~\\
			\hline
				\rule[-.3\baselineskip]{0pt}{3.2cm}~ & ~\\
			\hline
		\end{tabularx}
		
		\begin{tabularx}{\textwidth}{|X|}
			\hline
			Notes\\
			\hline
				\rule[-.3\baselineskip]{0pt}{3cm}~\\
			\hline
		\end{tabularx}

	\newpage
	We improve at teaching the same way that students learn---with practice and feedback. Reflection is a way
	to give ourselves feedback so that we can improve even if someone wasn't there to see our lesson.

	\item \textbf{Reflect.} 
	\begin{enumerate}
		\item What were one or two highlights/things that went well during the lesson?
			\vspace{\stretch{1}}
		\item What could be changed to improve this lesson next time? 
			\vspace{\stretch{1}}
		\item Did my lesson live up to my expectations? Why?
			\vspace{\stretch{1}}
	\end{enumerate}
\end{enumerate}

\newpage
\section*{Resources}
\subsection*{Learning Objectives Verbs}
When writing learning objectives you should use verbs that are specific and measurable. For example, 
the objective ``Students will \emph{understand} examples of\ldots'' is difficult to measure, but
``Students will \emph{produce} examples of\ldots'' can be measured by asking them to produce an example and
seeing if they can!

What follows is a list of verbs that can be used when building learning objectives. They are categorized by the levels of 
\href{https://en.wikipedia.org/wiki/Bloom\%27s_taxonomy}{Bloom's Taxonomy of Educational Objectives}.

\begin{enumerate}
	\item[\bfseries Knowledge] The successful student will recognize or recall learned information.

		\vspace{.1cm}
		~\begin{tabularx}{.7\textwidth}{|XXXX|}
			\hline
			list & record & underline & state\\
			define & arrange & name & relate\\
			describe & tell & recall & memorize\\
			recall & repeat & recognize & label\\
			select & reproduce &&\\\hline
		\end{tabularx}
		\vspace{.1cm}
	\item[\bfseries Comprehension] The successful student will restate or interpret information in their own words.

		\vspace{.1cm}
		~\begin{tabularx}{.7\textwidth}{|XXXX|}
			\hline
explain & describe & report & translate\\
express & summarize & identify & classify\\
discuss & restate & locate & compare\\
reiterate & review & illustrate & tell\\
critique & estimate & reference & interpret\\
			\hline
		\end{tabularx}
		\vspace{.1cm}
	\item[\bfseries Application] The successful student will use or apply the learned information.

		\vspace{.1cm}
		~\begin{tabularx}{.7\textwidth}{|XXXX|}
			\hline
apply & sketch & perform & use\\
solve & respond & practice & construct\\
role-play & demonstrate & conduct & execute\\
			complete & dramatize & employ&\\\hline
		\end{tabularx}
		\vspace{.1cm}
	\item[\bfseries Analysis] The successful student will examine the learned information critically.

		\vspace{.1cm}
		~\begin{tabularx}{.7\textwidth}{|XXXX|}
			\hline
	analyze & inspect & test & distinguish\\
categorize & critique & differentiate & catalogue\\
diagnose & appraise & quantify & extrapolate\\
calculate & measure & theorize & experiment\\
			relate & debate&&\\\hline
		\end{tabularx}
		\vspace{.1cm}
	\item[\bfseries Synthesis] The successful student will create new models using the learned information.

		\vspace{.1cm}
		~\begin{tabularx}{.7\textwidth}{|XXXX|}
			\hline
develop & revise & compose & plan\\
formulate & collect & build & propose\\
construct & create & establish & prepare\\
design & integrate & devise & organize\\
			modify & manage&&\\\hline
		\end{tabularx}
		\vspace{.1cm}
	\item[\bfseries Evaluation] The successful student will assess or judge the value of learned information.

		\vspace{.1cm}
		~\begin{tabularx}{.7\textwidth}{|XXXX|}
			\hline
review & appraise & choose & justify\\
argue & conclude & assess & rate\\
compare & defend & score & evaluate\\
			report on & select & interpret &
			investigate \\ measure & support&&\\\hline
		\end{tabularx}
		\vspace{.1cm}
\end{enumerate}

\newpage
\subsection*{Dee Fink's Situational Factors to Consider}

Before coming up with learning objectives, consider the various contexts in which
your lesson will occur by answering the following questions from Dee Fink.

\begin{enumerate}
	\item \textbf{Specific Context of the Teaching/Learning Situation}

How many students are in the class? Is the course lower division, upper division, or
graduate level? How long and frequent are the class meetings? How will the course be
delivered: live, online, or in a classroom or lab? What physical elements of the learning
environment will affect the class?

\item \textbf{General Context of the Learning Situation}

What learning expectations are placed on this course or curriculum by: the university,
college and/or department? the profession? society?

\item \textbf{Nature of the Subject}

Is this subject primarily theoretical, practical, or a combination? Is the subject primarily
convergent or divergent (e.g., are there multiple, persistent ``schools of thought'' or does the field
come to a consensus on a particular paradigm)? Are there important changes or controversies occurring within the
field?

\item \textbf{Characteristics of the Learners}

What is the life situation of the learners (e.g., working, family, professional goals)? What
prior knowledge, experiences, and initial feelings do students usually have about this
subject? What are their learning goals, expectations, and preferred learning styles?

\item \textbf{Characteristics of the Teacher}

What beliefs and values does the teacher have about teaching and learning? What is
his/her attitude toward: the subject? students? What level of knowledge or familiarity does
s/he have with this subject? What are his/her strengths in teaching?
\end{enumerate}

\subsection*{Dee Fink's Six Dimensions of Significant Learning}

Dee Fink's six dimensions of significant learning help break down the question:
\begin{quote}
	\emph{
		A year (or more) after this course is over, I want and hope that students will\ldots.}
\end{quote}

The dimensions\footnote{These aren't phrased exactly how Dee Fink phrased them.} are
\begin{itemize}
	\item[\textbf{Foundational}] What information/ideas are important to \textbf{understand and remember}? 
	\item[\textbf{Application}] What thinking/skills should a student \textbf{acquire}?
		Critical thinking, in which students analyze and evaluate? Creative thinking, in which students imagine and create?
		Practical thinking, in which students solve problems and make decisions?

		What important skills do students need to gain?

		Do students need to learn how to manage complex projects?

	\item[\textbf{Integration}] What ideas should students \textbf{connect} with their other courses/profession/life?

	\item[\textbf{Human}] What should students learn about themselves? How should students \textbf{interact} with each other? 

	\item[\textbf{Caring}] What changes do you hope for in the way students \textbf{feel}? In their interests? In their ideas?

	\item[\textbf{Meta}] Learning how to learn. What would you like students to learn about being a good student in a course like yours?
		About learning this particular subject? About how to become a self-directed learner?
\end{itemize}

\newpage
\subsection*{Sample Lesson Plan for MAT136\footnote{Essential questions and Objectives from Professor Mayes-Tang, 2019.}}

\paragraph{Essential Questions}
\begin{itemize}
	\item	Why is it possible for an infinite sum to equal a finite value?
	\item How do we add an infinite number of numbers?
	\item Why are problems easier to solve when they involve polynomials or
`infinite polynomials'?
	\item How can we represent arbitrary functions as infinite polynomials?
	\item How can tools from calculus be used to make sense of the infinite?
\end{itemize}

		\begin{tabularx}{\textwidth}{|s|X|s|}
			\hline
			Date & Title & Class Duration\\
			\hline
			%\rule[-.3\baselineskip]{0pt}{1cm}
			Jan. 1, 2022& Geometric Series & 50 min\\
			\hline
		\end{tabularx}
		
		\begin{tabularx}{\textwidth}{|X|}
			\hline
			Objectives\\
			\hline
			Students should be able to\ldots
			\begin{itemize}
				\item Define and identify geometric sequences, finite sums, and series
				\item Compute the sum of finite geometric series and infinite geometric
				series
				\item Quickly identify whether a geometric series converges or diverges
				\item Prove the formula for the $n$th partial sum of a geometric series by
				computing
				\item Translate descriptions of accumulating quantities into algebraic sums
				and series, including choosing appropriate indices
			\end{itemize}\\
			\hline
		\end{tabularx}
		
		\begin{tabularx}{\textwidth}{|c|X|}
			\hline
			Timestamp & Activity\\
			\hline
				12:10 (7min) & 
				\begin{minipage}{\linewidth}~\\
				\textbf{Think, Pair-share:} Slice the cake example. You brought a cake
				to a birthday party with you and your friends, but more people keep showing up. You keep slicing the remaining cake in half
				to make sure there's cake left for new arrivals.

				Discussion questions: will you need more than one cake to feed the guests? Will you use all of the cake? Write a formula
					for the size of the $n$th slice.
					\\[-.1cm]
				\end{minipage}
				\\
			\hline
				12:17 (4min) & \textbf{Mini-lecture:} Summation-notation review.\\
			\hline
				12:21 (5min)& \textbf{Group work:} Write a formula \emph{using summation notation} for the
				total amount of cake given out after the $n$th slice is given out.\\
			\hline
				12:26 (15min) & 
				\begin{minipage}{\linewidth}~\\
					\textbf{Class discussion:} Get several examples of summation formulas using \emph{different index variables}
					on the board; make sure some start at index $0$ and others at index $1$; make sure at least one is incorrect.

					Ask students to discuss with neighbour which examples are correct/incorrect.
					\\[-.1cm]
				\end{minipage}
				\\
			\hline
				 12:31 (4min) & \textbf{Short group work:} Ask the class what for a formula for how much cake \emph{remains} after the $n$th
				 piece is distributed. Have them work with their neighbour to use this information to find a closed-form representation
				 of their summation formula.\\
			\hline
				 12:35 (4min) & \textbf{Student Checkin:} Ask how students are feeling about the work so far and if anyone
				 has seen the types of formulas we've come up with before.\\
			\hline
				 12:39 (15min) & \textbf{Guided worksheet:} Project the worksheet that builds up to the formula for a
 				partial and infinite geometric series. Have students work through the worksheet in small groups.\\
			\hline
				 12:54 (6min) & \textbf{Mini-lecture/discussion:} Dialog with the class how their new formula can be used
				 to answer the Cake question. Ask what happens when $r\geq 1$ is plugged in the formula they got.
				 Emphasize the need to do ``sanity checks'' in math even when you've derived a formula.\\
			\hline
		\end{tabularx}
		
		\begin{tabularx}{\textwidth}{|X|}
			\hline
			Notes\\
			\hline
				The important part of this lecture is that students understand that they can \textbf{solve problems
				themselves by thinking}. That is, the formula at the end should be de-emphasized. If students
				come to class having already memorized
				the formula, encourage them to solve the Cake problem without the formula.\\
			\hline
		\end{tabularx}

\newpage
\subsection*{Teaching Toolbox}
Below is a list of teaching activities taken from the U of T Calculus Community of Practice Guidebook.
\begin{multicols}{2}
%% \addtoolboxitem{Title}{What is it?}{Variants}{Uses}

\addtoolboxitem{Timed Think}
{Pose a question, give the students a set time 
to think about the answer silently, then ask the question again and 
call on a student to answer it.}
{Timed group discussion, timed pair
discussion, untimed lecture pause (ie, pause lecture until enough
hands are raised).}
{In the case of a silent class, this will generally
force an answer from the room. In feedback, students often want
more time to think about a problem after it is asked.}	


\addtoolboxitem{Think-Pair-Share (TPS)}
{With a question posed, students 
think silently about an answer. Then they pair up, and share their 
response with their neighbour.}
{Explain instead of share, 
Think-Group-Share, Think-Pair-Share 
with a classroom voting system (Think-Vote-Share-Vote), 
repeated Think-Pair-Share-Think-Pair-Share, Think-Pair-Share where during the "Pair" step the student must find a classmate who has a different answer to them.}
{This is the crux of active learning, 
and should be used for simple, conceptual questions, not 
long computational question.}

\addtoolboxitem{Punctuated Lecture}
{A fast-paced lecture, where at the end of 
every slide there is a comprehension-style question, such as 
\enquote{why does this computation work?}, \enquote{explain 
this step}. The question is then worked through using 
\textbf{TPS}.}
{Questions can be posed in an interactive voting system, see also:
\textbf{Going Through the Steps}}
{This is a good technique for getting 
through material that is emphasized in class, as opposed 
to reviewing pre-class material. It combos well with 
timed thinks and TPS.}

\addtoolboxitem{Free-For-All Online Discussion}
{A class-wide open discussion, where 
any student can contribute text or pictures to a forum-style 
tool. This can be done using TopHat, Google Docs, or Quercus.}
{Timed responses, group submission, 
see also: \textbf{Write \& Quiz}, \textbf{Write the Test}, 
and \textbf{1-Minute Essay}.}
{This can be used for getting a lot of ideas 
fast, and to consolidate student solutions for everyone 
to see and use later in studying.}

\addtoolboxitem{1-Minute Essay}
{Students write a 1-minute essay 
linking concepts, explaining a concept, or summarizing 
a concept learned in class. This can be done with 
or without a specific prompt.}
{5-minute essay, 1-minute paragraph, 
writing in groups, 1-minute list, end-of-class summary. See 
also: \textbf{ice cream sandwich}, \textbf{write \& quiz}, 
and \textbf{free-for-all discussion}.}
{This activity is a good conclusion to a 
lecture, module, or other topic. Walking around to pick 
students to read their sentences out loud to their 
neighbours can also be used to build comfort 
within a group.}

\addtoolboxitem{Fill-In Blanks}
{A short, fill-in-the-blanks 
pop-quiz, either on the board, on a slide, or through 
an interactive classroom response system.}
{Giving the word possibilities, see also: \textbf{1-minute essay}}
{This is a good activity to start a 
class or a topic, and to make sure that everyone 
has done the reading, is on the same page, 
and is ready to start learning.}

\addtoolboxitem{Write \& Quiz}
{Students come up with a question, 
then partner up and quiz each other.}
{Larger groups sharing the questions, forcing questions to be
conceptual or computational. See also: \textbf{free-for-all
discussion}, \textbf{write the test}, \textbf{paper passaround}.}
{This is a good activity when there is a lot 
of relatively straightforward material that would take a long 
time to go over, but should be spot-checked. This activity 
also gets students to think about what questions could appear 
on tests.}

\addtoolboxitem{Concept Map}
{A concept map is a directed 
or undirected graph with concepts for nodes and 
edges for connections between them. Edges should be labeled. 
The activity is to make a concept map.}
{Students can get: list of concepts, 
the concept map without edges, or the concept map 
with unlabelled edges.}
{This is an excellent review-session 
tool, as it takes up the entire class, and can cover a 
lot of material. The main goal of the activity is to 
introduce students to the idea, not to finish the map.}

\addtoolboxitem{Critique History \& News}
{A short lesson on the origins 
of the course content, or a news clipping related 
to it. The more primary sources, the better.}
{Having students see the 
primary source and critique it.}
{Some students love this, some hate 
it, but it can be used to show to students that math 
was always difficult, and that other people also 
make mistakes.}

\addtoolboxitem{Going Through the Steps}
{Write a sequence of steps to solve 
a problem, and go through them one by one using other 
teaching-toolbox tools.}
{Handout with steps, allow 
students to find the correct order of the steps. 
See also: \textbf{punctuated lecture}, 
\textbf{think-pair-share}.}
{This is a good activity to use 
to teach students a specific problem-solving strategy.}

\addtoolboxitem{The Tommy Question}
{Tommy writes an incorrect or incomplete solution to a problem, 
and the students need to fix the solution.}
{Use previous exam solutions, have students grade the response, 
let students address their explanation to Tommy}
{This is a good activity to target subtle misconceptions and 
to highlight common pitfalls students may encounter}

\addtoolboxitem{Round Robin}{Students get into groups and take turns
highlighting key points from the lecture or course content. Possible
variant: \textbf{Playing Darts}---Students shout concepts related to
the material, and the
instructor writes them down on the board. Finish with a follow-up at the end of class outlining what was and wasn't covered in class. 

Pairs well with a \textbf{timed think}}
{This activity can be an opener and a closer, and reminds 
students that they are also responsible for things not covered 
in class.}{This is a good activity to use during review sessions when drumming up concepts from the past is important.}

\addtoolboxitem{Ice Cream Sandwich}
%(AKA: neapolitan ice-cream)
{Students write three things: something they've 
mastered in the unit, something they're struggling with, and something that was cleared up}{Any triple of questions can work!}{Student reflection, seeing learning as a dynamic process.}

\addtoolboxitem{Draw the Definition/Theorem}{Provide the students with a definition or theorem and have them illustrate this definition or theorem. Their drawing could be of an explicit example which works, or something more general.}{Provide the students with a definition or theorem and a sketch. Have students fill in anything which is missing, and have them colour-code parts of the definition or theorem and colour the corresponding part of the picture that colour as well.}{Allows students to engage with a definition or theorem and build some intuition about the concept.}

\addtoolboxitem{Geogebra Applets}{Geogebra has many free applets available on its platform. It is possible to find many applets which are interactive and allow students to interact with different concepts. Be sure to test out the applet before use to ensure it is what you're looking for!}{There are a variety of different applets available. Some of the applets can be an entire activity on their own, and others can be a supplement to help you illustrate an idea to students.}{Allows students to build intuition behind different concepts in a visual, yet interactive way.}

\addtoolboxitem{Jigsaw}{Break the class into groups and have each group solve a different part of a problem. At the end, every one comes together to synthesize their solutions to solve the main problem.}{This can be used to fill in tables, to see patterns emerging through examples, or to solve a larger problem as a group.}{This can be used to explore theorems or definitions and build intuition.}

\addtoolboxitem{Gameshow}{For review, have large a variety of questions prepared which you will display one at a time. Students will answer questions, and will keep track of their longest streak of correct answers.}{Could have the class break up into teams to play, or change the format to mimic a real game show (for example, Jeopardy).}{A fun way to engage students during a review class before a midterm or final exam.}

\addtoolboxitem{Paper Passaround}{In groups, students write things on papers, and exchange with other groups. The other group reads, and responds on the page.}{Some examples: each group solves one of three problems on the board, and then critiques another group's solution. See also: \textbf{Write \& Quiz}}{This is a good physical writing exercise that 
lets students see and critique how math can be communicated in writing.}

\addtoolboxitem{Run the Test}{Students play the role of the instructor in writing, grading, or helping students during a test-environment. For example, students may be given a solved test to critique, or could be asked to give advice to students before, after, or during the test (see: \textbf{Tommy question})}{An online tool to collect student-made test questions to create a question bank can be useful.}{This exercise can be used to demystify the test, and to highlight common mistakes. See also: \textbf{Write \& Quiz}, \textbf{Free-For-All Online Discussion}}

\end{multicols}

\subsection*{Additional Resources}
Additional resources (including a copy of this document) can be found at \url{https://github.com/siefkenj/teaching-resources}.

\begin{center}
	\qrcode[height=1in]{https://github.com/siefkenj/teaching-resources}
\end{center}
		
\end{document}
